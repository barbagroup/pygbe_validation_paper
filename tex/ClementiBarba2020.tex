\documentclass{rstransa} %%%%where rstrans is the template name

\usepackage{amsfonts}
\usepackage{amsmath}
\usepackage{amssymb}
\usepackage{booktabs}
\usepackage{caption}
\usepackage{color}
\usepackage{comment}
\usepackage{float}
\usepackage{graphicx}
\usepackage{hyperref}
\usepackage[utf8]{inputenc} % allows using accents directly in text, like ÔøΩ
\usepackage{subfig}
\usepackage{xspace}
\usepackage{anyfontsize} % added to avoid font size replacement
\usepackage{multirow} 
\usepackage{paralist}       % compactitem environment

\newcommand{\pygbe}{\texttt{PyGBe}\xspace}

\graphicspath{{figs/}} %  PATH to figure files-- change to ./ for submission


\begin{document}
%%%% Article title to be placed here
\title{Reproducible Validation and Replication Studies in Nanoscale Physics}

\author{%%%% Author details
N. C. Clementi$^{1}$, L. A. Barba$^{1}$}

%%%%%%%%% Insert author address here
\address{$^{1}$Department of Mechanical and Aerospace Engineering, 
The George Washington University, Washington D.C., USA }

%%%% Subject entries to be placed here %%%%
\subject{validation and verification, reproducibility and replication, computational modeling, computational physics, research software}

%%%% Keyword entries to be placed here %%%%
\keywords{reproducibility, validation, replication, nanophysics}

%%%% Insert corresponding author and its email address}
\corres{L.A. Barba\\
\email{labarba@gwu.edu}}

%%%% Abstract text to be placed here %%%%%%%%%%%%
\begin{abstract}
    The abstract text goes here. 
\end{abstract}
%%%%%%%%%%%%%%%%%%%%%%%%%%%
    
%%%%%%%%%% Insert the texts which can accommodate on firstpage in the tag "fmtext" %%%%%
    
\begin{fmtext}  
\section{Introduction}

Some fields of research, particularly solid mechanics and computational fluid dynamics, have a long tradition of community consensus building and established practices for verification and validation of computational models. 
Such practices are uncommon in other fields of science, especially if they have more recently become computationally intensive.
Verification and validation also become increasingly difficult when the computational models arise from many levels of mathematical and physical modeling, representing a complex system. 
In recent years, science as a whole has come to be concerned with reproducibility and replication as a new front in the continual campaign to build confidence on published findings. 
Together with formal processes of uncertainty quantification, we have now three complementary ``axes'' to building trust in science. 



\end{fmtext}

\maketitle
% Body of paper.

% continue Introduction

The lengths to which research communities should go to conduct activities in verification and validation, reproducibility and replication, and uncertainty quantification, are highly debated. 
Some journals require articles reporting on computational results to include proof of these activities, while most do not consider these aspects at all in their review criteria. 
In this paper, we tackle a sub-field of computational physics where tradition for these confidence-building activities is scant. 
The physical setting, excitation of resonance modes in nanostructures under an electromagnetic field, relies on multiple levels of modeling, while the experimental methods are complicated by the small length scales. 
We previously developed a computational model and software (called \pygbe) that has undergone code and solution verification, but a validation opportunity had remained elusive. 
Here, we present replication studies and a validation case based on published simulation and experimental results. 
Moreover, the studies in this paper were conducted under rigorous reproducibility practices, and all digital artifacts needed to reproduce every figure are shared in reproducibility packages available in a GitHub repository and archival services. 

\section{Background and methods}\label{sec:background}
%!TEX root = ClementiBarba2020.tex

\subsection{Verification, validation, reproducibility and replication}

\subsection{Description of the PyGBe software}

\subsection{Physics context for this work}

\section{Results} \label{sec:results}
%!TEX root = ClementiBarba2020.tex

\subsection{Replication of results from Rockstuhl, et al., 2005}

%%% Rockstuhl work summary%%%%
The work of Rockstuhl et al.\cite{rockstuhl2005} studies the phonon-polariton response of silicon carbide (SiC)
nanoparticles using the boundary element method. They analyze different "cylindrical particles"
made of 6H-SiC as a function of geometrical cross section.They use a 2D boundary element method 
previously developed by the group \cite{rockstuhl2003}, where they extend third dimension to infinity
(hence the name cylinder).
%%% Rockstuhl work summary%%%%

We decided to replicate one of the results presented on Fig.14 of their paper \cite{rockstuhl2005}, which 
presents the scattering cross section of a SiC rectangular cylinder for three different aspect ratios. 
To comply with our quasistatic approach ($\lambda > d$ where $d$ is the characteristic
dimension of the geometry) we chose the particular case where a=672 nm and b=328 nm.

\textbf{Difference between methods, mesh and dielectric data}

\textit{method}: The main difference between Rockstuhl simulations and ours is that they solve a 2D problem of 
full Maxwell equations, while we solve a 3D problem with the electrostatic approximation . 
We do not count with information on their code implementations, discretization schemes, or solver.  
We compute extinction cross section (scattering plus absorption) while Rockstuhl work present only scattering
cross section. But since  the variable of interest is the wavelength at which the resonance modes occur, the 
results are comparable.

\textit{mesh}: Their work does not provide any details regarding the discretization of the geometries or 
parameters involved in the simulations.
We performed a grid independence study to make sure that we are solving properly the equations and 
minimizing errors due to discretization. Rockstuhl has a 2D geometry (infinite third dimension) while we deal 
with a 3D representation of the geometry. 

\textit{dielectric data}: They use 6H-SiC as material, and they obtained their data from a a source that we
were not able to replicate. Instead we are using experimental data for 4h-SiC that was provided to us 
via private communications from the authors of Ellis et al \cite{ellis2016}.  

\subsubsection{Grid independance study}

We performed a grid independence study on a SiC cube of side $L=535$ nm submerged in air, under a 
constant electric field aligned with the z-axis (similar setup to quadratic cylinder on Fig 18 of 
Rockstuhl et.al)\cite{rockstuhl2005}. 
Due to the nature of the geometry and its sharp edges it was challenging to see proper convergence. However,
for this type of physics, convergence has been studied in our previous work \cite{ClementiETal2019}. (NOT SURE IF WE NEED TO MENTION THIS)
Figure \ref{fig:cube535} shows the grid independence study where we go from a mesh of 15552 
triangles (density = 1.11x10$^-4$ $N/\text{\AA}$) to 19200 (density = 9.05x10$^-5$ $N/\text{\AA}$) 
triangles, and the computed results do not change.

\begin{figure}
    \centering
    \includegraphics[width=0.85\textwidth]{cubeL535nm_15Kvs19K.pdf} 
    \caption{Grid independence study for a SiC cube of side $L=535 nm$ submerged in air under a constant 
    electric field in the z-direction. The curves represent the extinction cross section divided by $L^2$ 
    as a function of wavelength divided by $L$}
    \label{fig:cube535}
 \end{figure}

It is worth noting that the extinction cross section curve in Figure \ref{fig:cube535} has extra peaks 
compared to the results of Figure 18 on Rockstuhl, this is due to the 3D nature of our case and the sharp 
edges, being the latest an effect also mentioned in Rockstuhl et al. results. The 3D effect will be approach
in the following results where we attempt a replication of one of teh results of Figure 14 of Rockstuhl work. 

\subsubsection{Replication of Figure 14 (case a1) of Rockstuhl 2005}

We chose to replicate the result of Rocksuthl et al. presented in Figure 14. In particular the case where $a=672$ nm 
and $b=328$ nmn, since these dimension comply with the requirements of the quasistatic approximation.
They present the normalized scattering cross section of a SiC rectangular cylinder, and they perform simulation 
for two different setups. In Rockstuhl Figure 14 (left) they have the wave vector (illumination) along the long 
side of the geometry, which means that the electric field is parallel to the short side of the rectangle. Following, 
a similar analysis Rockstuhl Figure 14 (right) they have the wave vector (illumination) along the short
side of the geometry, which means that the electric field is parallel to the long side of the rectangle. These orientations,
correspond to the images (A) and (B) in the sketch presented in Figure \ref{fig:rectangle_sketch}.

Based on the results of the grid independence stud we decided to we used the density of the cube as a reference to 
produce the meshes for the replication of figures 14a and 14b on Rockstuhl's work (case a1=672, b=328). For the third 
dimension we needed to chose a value that represents "infinity". To achieve this, we perform a study on the effects of 
elongate the third dimension. In Figure \ref{fig:ext_y_14} we present the results for two different values of the 
third dimension, $y=1344$ nm (2xa) and $y=2688$ nm (4xa). 
We see that as we extend the third dimension the intensity of some peaks decreases, this is because some
of the peaks are related to the third dimension. Therefore, from now on we use $y=y=2688$ nm. It is worth 
noting that we could have increased this value more, however, the simulations will become computational
more expensive and we will still have the effects of having a 3D model.

\begin{figure}
    \centering
    \includegraphics[width=0.45\textwidth]{rockstuhl_rectangles.pdf} 
    \caption{Rockstuhl runs configurations}
    \label{fig:rectangle_sketch}
\end{figure}

\begin{figure}
    \centering
    \subfloat{\includegraphics[width=0.45\textwidth]{ext_y_14a.pdf}}
    \subfloat{\includegraphics[width=0.45\textwidth]{ext_y_14b.pdf}} 
    \caption{Effect of the elongation of the third direction ($y$) on the 
        extinction cross section of a rectangular prism of SiC of dimensions $a=672$ nm 
        and $b=328$ nm, submerged in air and under a constant electric field 
        parallel to the z-axis. The left plot corresponds to a configuration such that the electric 
        field is parallel to $b$ (configuration (A) on Figure \ref{fig:rectangle_sketch}), while the 
        right plot corresponds to a configuration such that the electric field is 
        parallel to $a$ (configuration (B) on Figure \ref{fig:rectangle_sketch}}
    \label{fig:ext_y_14}   
 \end{figure}


To generate the meshes for these simulations, we used the open source software Trimesh 
(\url{https://github.com/mikedh/trimesh}), but we realized that it was not producing a 
uniform mesh and that it was not possible to produce regular triangles with the functions 
available. To overcome this, we created our own mesh using python scripts, and therefore 
obtain uniform meshes. We wanted to study the effect of a uniform mesh as well as the effect
on rounding the edges. We wanted to round the edges since in Rockstuhl work this was mentioned 
as a factor that will introduce extra peaks on the response. We were not able to control the 
roundness as a function of arc of curvature or the dimensions of the rectangular prism, so we 
decided to use the default settings on Trimesh. Figure \ref{fig:tri_reg_round_14} shows the 
results on the effect of uniformity and roundness. You can see that the second peak is not
present in the green curve, which can be attributed to the effect of the roundness. This is 
consistent with the results on Rockstuhl's work. 

\begin{figure}
    \centering
    \subfloat{\includegraphics[width=0.45\textwidth]{tri_reg_round_14a.pdf}}
    \subfloat{\includegraphics[width=0.45\textwidth]{tri_reg_round_14b.pdf}}
    \caption{ Effect of uniformity and roundness on the mesh on the 
    extinction cross section of a rectangular prism of SiC of dimensions $a=672$ nm, 
    $b=328$ nm and $y=2688$ nm, submerged in air and under a constant electric field 
    parallel to the z-axis}
    \label{fig:tri_reg_round_14}
 \end{figure}

Once we have found the "best" possible geometry approximation, we show how our results 
(green curve in Figure \ref{fig:tri_reg_round_14}) compare 
with the original results from Rockstuhl's Figure 14. We have obtained Rockstuhl's curves by using 
the WebPlotDigitizer (\url{https://apps.automeris.io/wpd/}). The replication results are 
presented in Figure \ref{fig:rep_14}. We can say that the main peaks presented in Rockstuhl et al. are 
successfully replicated. While we still have the presence of a third peak in our results, we believe 
this is a consequence of the 3D nature of our geometry.

 \begin{figure}
    \centering
    \subfloat{\includegraphics[width=0.45\textwidth]{replication_14a.pdf}}
    \subfloat{\includegraphics[width=0.45\textwidth]{replication_14b.pdf}} 
    \caption{Replication of Figure 14 from Rockstuhl 2005. Extinction cross section of a
    rectangular prism of SiC of dimensions $a=672$ nm, $b=328$ nm and $y=2688$ nm, submerged
    in air and under a constant electric field parallel to the z-axis.}
    \label{fig:rep_14}
 \end{figure}


 \subsection{Replication of results from Ellis et al., 2016, and validation}

 %%% Ellis work summary %%%%
The work of Ellis and coworkers \cite{ellis2016} study the aspect-ratio evolution of high-order
resonant modes in localized surface phonon-polariton nanostructures. They study the
excitation of multipolar localized surface phonon polaritons (SPhP) resonances, by measuring
polarized reflectance measurements on 4H-SiC pillars of fixed height ($H=950$ nm), fixed 
width ($W=400$ nm) and varied length ($L=400-4800$ nm) and therefore an aspect ratio
($AR=L/W=1-12$). These pillars are patterned on a squared grid with a pitch $P=L+500$ nm
to reduce coupling. In their simulations and experiments they excite the localized SPhP
resonances, they perform polarized reflectance measurements with the incident polarization 
oriented parallel or perpendicular to the long axis of the pillars.  
 %%% Ellis work summary %%%%

We aim to replicate the computational result presented in Figure S4 of their supplementary 
material (NOT SURE HOW TO CITE THIS) that correspond to the black curve on their plot. In this 
figure they show simulation results for the resonance spectral position of the lower frequency 
mode when having parallel polarization ($E^{\parallel}_{100}$), with an incidence angle of 22$^\circ$.
We first attempted to replicate this result since they have simulations with the gap between pillars 
is 5000 nm, and this cause negligible coupling. This result is suitable to compare with 
what we can model using \pygbe.  




\textbf{Difference between methods, mesh and dielectric data}

\textit{methods}
In Ellis et al simulations they solve full Maxwell equations via the RF package of the finite
element method software COMSOL. To represent the array of pillars and their interactions, they use
periodic boundary conditions of one pillar over a portion of substrate. They present the reflectance 
as a function of the wave number. As we mentioned before, we use the boundary element method in 
the quasistatic approximation, which is suitable in these case since Ellis pillar's size is small 
compared to the wavelength involved in the simulations (800-1000 cm$^-1$). We measure extinction cross 
section, which will express as peaks instead of inverted peaks (as in reflection) since you see maximum
extinction when there is minimum reflection. The intensity of the peaks is not comparable, but we are 
looking to match the wave number at which these events happen. 

\textit{mesh}
For the case of aspect ratio AR=4 we count with a triangular surface mesh provided by the authors of 
Ellis et al. Which is a non-structured mesh of approximately 4 thousand elements. To produce the remaining 
meshes, we use our python script to generate a uniform mesh and round the edges using Trimesh. 
After doing certain refinement studies we noticed that to achieve close results to the original mesh, using a
uniform mesh, we needed to double the density (I HAVE SOME PLOTS ON THIS RESULTS BUT 
I'M NOT SURE IF WE WANT TO PUT THEM, SINCE WE DON'T HAVE SPACE).
To replicate the results of figure S4 on their supplementary material, we use uniform meshes. While for the 
replication of figure 2a and the validation result we used the mesh provided by the authors of Ellis et al. 


\textit{dielectric data}
The dielectric data for the simulations, was given by the authors of the paper via a private communication. 

To be able to replicate this results, we need to identify the lower frequency mode for each of
the aspect ratios in our computations. For each aspect ratio (AR) 
value from 1 to 7 we computed the extinction cross section $C_{ext}$ across the wave number in the range
800-1000 $cm^{-1}$. We identify the lower frequency mode ($E^{\parallel}_{100}$ in Ellis paper)  that is not a 
longitudinal mode (mode related to the hight of the pillar, only visible when the incidence of the 
electric field is not normal). We can identify the longitudinal modes by comparing normal vs 22 degrees 
angle of incidence runs (see Figure \ref{fig:ellis_ang_inc}). As we mentioned the longitudinal modes, due to its nature, do not appear when we 
have the normal incidence . Then, on the 22 degree incidence computations we choose the lower wave number mode
that it's not a longitudinal mode. 

Note: Ellis et al., changes the angle of incidence of the illuminating vector while in \pygbe we achieve this by
rotating the geometry instead. 

\begin{figure}
    \centering
    \includegraphics[width=0.45\textwidth]{ellis_ang_inc.pdf} 
    \caption{Diagram of angles of incidence in \pygbe setup to comply with Ellis et al. disposition.}
    \label{fig:ellis_ang_inc}
 \end{figure}

Figure \ref{fig:AR_22_vs_norm} show the results of the extinction cross section of a SiC pillar of fixed
height ($H=950$ nm), fixed width ($W=400$) and varied length ($L=400-2800$ nm). The simulations where performed 
performed for the Long Edge orientation, meaning that the electric field is aligned with the length pillar 
when having normal incidence (see Figure \ref{fig:ellis_ang_inc}). 

In Figure \ref{fig:AR_22_vs_norm} and Table \ref{tab:ar_peaks} we show the resonance peaks and
their corresponding wavelengths. In the table we mark in bold the peaks which correspond to 
the $E^{\parallel}_{100}$ mode. 

\begin{figure}
    \centering
    \includegraphics[width=0.8\textwidth]{AR_22_vs_norm.pdf} 
    \caption{Extinction cross section across wave number for SiC pillars of multiple aspect ratios,  
             (H=950 nm, W=400nm, L=400-2800 nm (AR=1-7)), where we have normal incidence and a 
             22 degrees incidence.
            }
    \label{fig:AR_22_vs_norm}
 \end{figure}


\begin{table}
    \begin{center}
      \caption{Wavelength at which peaks happen for different aspect ratios, for runs where the electric
      field is parallel to the length (L) of the pillar. We have normal incidence and 22 degrees.}
      \label{tab:ar_peaks}
      \begin{tabular}{c c c c c c c c}
        \textbf{AR} \\
        \hline
        \multirow{2}{*}{1} & $\perp$ & \textbf{917.73} & 934.092 & 949.604 & 957.325 \\ % <-- Combining 2 rows with arbitrary with (*) and content 12
        & 22$^{\circ}$ & 883.926 & \textbf{917.73} & 935.052 & 949.604 & 957.325 \\ % <-- Content of first column omitted.
        \hline
        \multirow{2}{*}{2} & $\perp$ & \textbf{903.233} & 926.395 & 944.762 & 958.242 \\ % <-- Combining 2 rows with arbitrary with (*) and content 12
        & 22$^{\circ}$ & 896.517 & \textbf{903.233} & 926.395 & 944.762 & 958.242 \\ % <-- Content of first column omitted.
        \hline
        \multirow{2}{*}{3} & $\perp$ & \textbf{888.793} & 922.552 & 931.223 & 948.613 & 958.242 \\ % <-- Combining 2 rows with arbitrary with (*) and content 12
        & 22$^{\circ}$ & \textbf{888.793} & 899.418 & 922.552 & 931.223 & 958.242 \\ % <-- Content of first column omitted.
        \hline
        \multirow{2}{*}{4} & $\perp$ & \textbf{876.186} & 929.32 & 946.639 & 958.242 \\ % <-- Combining 2 rows with arbitrary with (*) and content 12
        & 22$^{\circ}$ & \textbf{876.186} & 901.281 & 921.618 & 929.32 & 945.745 & 958.242 \\ % <-- Content of first column omitted.
        \hline
        \multirow{2}{*}{5} & $\perp$ & \textbf{865.576} & 926.395 & 945.745 & 958.242 \\ % <-- Combining 2 rows with arbitrary with (*) and content 12
        & 22$^{\circ}$ & \textbf{865.576} & 901.281 & 921.618 & 958.242 \\ % <-- Content of first column omitted.
        \hline
        \multirow{2}{*}{6} & $\perp$ &  \textbf{856.904} & 914.793 & 923.489 & 929.32 & 946.639 & 958.242\\ % <-- Combining 2 rows with arbitrary with (*) and content 12
        & 22$^{\circ}$ & \textbf{856.904} & 901.281 & 920.6 & 958.242\\ % <-- Content of first column omitted.
        \hline
        \multirow{2}{*}{7} & $\perp$ &  \textbf{850.134} & 910.963 & 921.618 & 928.372 & 946.639 & 958.242 \\ % <-- Combining 2 rows with arbitrary with (*) and content 12
        & 22$^{\circ}$ & \textbf{850.134} & 901.281 & 910.963 & 920.6 & 958.242\\ % <-- Content of first column omitted.
        \hline
      \end{tabular}
    \end{center}
  \end{table}

Once we identify the proper modes, we proceed to the replication of Figure S4 on the 
supplementary material of Ellis et al work. It is worth mentioning that we limit ourselves to
the replication of the black curve in their plot, since in this case the pillars are separated 
by 5000 nm which means the coupling effects are negligible. Figure \ref{fig:rep_FS4_ellis} shows
the results from  Ellis et al (digitized using the WebPlotDigitizer) and the results obtained
with \pygbe, and Table \ref{tab:err_AR} shows that the percentage error is below 2$\%$ for all
the cases


\begin{figure}
    \centering
    \includegraphics[width=0.85\textwidth]{AR_rep_FS4_Ellis2016.pdf} 
    \caption{Replication of figure S4 of supplementary material of Ellis 2016. Wave
    number at which happens the $E^{\parallel}_{100}$ mode for different aspect ratios.}
    \label{fig:rep_FS4_ellis}
 \end{figure}
 
 \begin{table}
    \centering
    \caption{Percentage error for different aspect ratios.} 
    \label{tab:err_AR}
    \begin{tabular}{c c}
    \hline%\toprule
    AR & \% error \\
    \hline%\midrule
     $1$ & $0.95$ \\
     $2$ & $0.67$ \\
     $3$ & $0.35$ \\
     $4$ & $0.16$ \\
     $5$ & $0.72$ \\
     $6$ & $1.20$ \\
     $7$ & $1.59$ \\
    \hline%\bottomrule
    \end{tabular}
\end{table}

\subsubsection{Validation of \pygbe against Fig 2a experimental results}

For the case of aspect ratio four (AR=4) we obtain the smallest error using the uniform
mesh. Since we count with the mesh provided by the authors, and knowing that our computation 
for the mode $E^{\parallel}_{100}$ compares well with their computations, we attempted to 
validate our simulations with their experimental results (red curve on paper), as well 
as to replicate their simulations (green curve on paper) on Figure 2a of their paper.
Figure 2a of Ellis et al, presents measured and simulated reflectance of SiC pillar
arrays with a 500 nm gap. All their measurements and simulations were performed with 
22$^\circ$ off-normal angle of incidence and incoming polarization parallel to the 
elongated size of the pillar. 

Using \pygbe we compute the extinction cross section of an isolated SiC pillar (AR=4)
with no substrate, submerged in air under a constant electric field in the z-direction, 
and rotate the orientation of the pillar to match the angle of incidence (see Figure \ref{fig:ellis_ang_inc}).
In Figure \ref{fig:pygbe_vs_exp_2a} we present 
comparison of our simulations and the experimental results from Ellis et al. There is a 
difference in the peaks wavelength that is noticeable. This could be attributed to the
fact that in their experiments the separation between pillars is of 500 nm, which implies
there are coupling effects that in our simulations are not contemplated.  

\begin{figure}
    \centering
    \includegraphics[width=0.85\textwidth]{pygbe_vs_exp_fig2a_Ellis.pdf} 
    \caption{\pygbe vs experiments of figure 2a of Ellis 2016. Ellis data 
    was digitized with web digitizer.}
    \label{fig:pygbe_vs_exp_2a}
 \end{figure}

\paragraph{First order correction}

Since our simulations do not count for coupling effects, we can not strictly imitate 
the conditions to validate our solver. However, from Figure S4 on the supplementary 
material of Ellis et al, we know that coupling effects affect the $E^{\parallel}_{100}$ 
mode by a shift of 12.17 cm$^-1$. Therefore as a \textit{first order correction} we can 
subtract this value from our simulations to count for coupling effects. In Figure 
\ref{fig:val_2a} we present the result after applying the correction for coupling effects.
It is worth mentioning that the far left (837 cm$^-1$) and far right (964 cm$^-1$) peaks 
on Ellis results, are peaks associated with zone-folded LO (longitudinal) phonons, in 
particular P1 (837 cm$^-1$) is not infrared active \cite{ellis2016}. Ellis and coworkers, 
do not discuss this result into detail, but they concentrate their analysis on the peaks
that happen between 864–961 cm$^-1$ 

\begin{figure}
    \centering
    \includegraphics[width=0.85\textwidth]{validation_FOA_fig2a_Ellis.pdf} 
    \caption{Validation against experiments in figure 2a of Ellis 2016, using first order approximation}
    \label{fig:val_2a}
 \end{figure}

If we use the same approximation and compare the results with Ellis simulations on
Figure 2 of their paper (green curve) we get (see Figure \ref{fig:rep_2a})

\begin{figure}
    \centering
    \includegraphics[width=0.85\textwidth]{replication_FOA_fig2a_Ellis.pdf} 
    \caption{Replication of simulations in figure 2a of Ellis 2016, using first
     order approximation}
    \label{fig:rep_2a}
 \end{figure}


\section{Discussion}\label{sec:discussion}
%!TEX root = ClementiBarba2020.tex

We have presented several results that replicate previous published findings in the general area of nanostructure responses to electromagnetic waves. 
Our field of interest is computational nanoplasmonics for applications in biosensors, and in a previous publication we developed the mathematical formulation and reported both solution verification activities, and an application demo with our software \pygbe, extended to treat complex dielectrics and imposed electric fields \cite{ClementiETal2019}.
The search for a physical context and published results that would allow us to undertake validation studies with \pygbe is what led us to this work. 
Even if we finally do have validation and replication cases, neatly presented here, the path to obtain these results was nonlinear, iterative, and arduous.

In the first case, we sought to replicate a result from Rockstuhl et al., 2005 \cite{rockstuhl2005}, where they computed the scattering cross-section as a function of wavelength for a silicon carbide rectangular nanostructure. 
They present their results as a plot (Figure 14 in their paper), and report the numeric value of the resonance wavelengths in the text. 
Lacking access to the secondary data behind the plots---computed from two-dimensional simulations with a boundary element solver---we were forced to manually digitize the values from the figure.
Our results are presented in Figure \ref{fig:rep_14}, together with the curve we obtain from digitizing the source image. 
We were successful at replicating the strong peaks reported  by Rockstuhl et al.\ at wavelengths
10.42 $\mu$m and 10.7 $\mu$ m,  when the electric field $E$ is parallel to short side of the rectangle, and 10.42 $\mu$m and 10.82 $\mu$ m
when $E$ is parallel to the long side. Our results contain extra (small) peaks that are not present in the work of Rockstul et al.
The first one, located between the main two peaks, we attribute to the the effect of
sharp edges (see Figure \ref{fig:tri_reg_round_14}), as when we introduce a level of roundness, it diminishes. The second extra peak is the far right one, and we believe this
peak is a consequence of the 3D nature of our geometry; as observed in Figure \ref{fig:ext_y_14} this peak intensity
decreases as the third dimension of the geometry lengthens.
The quantity of interest in these findings is the wavelength of the resonance peaks, and our results do indeed match the findings.

The second replication cases come from a paper by Ellis and co-workers \cite{ellis2016} studying the effect of aspect ratio on the excitation of high-order modes in localized surface phonon-polariton nanostructures. 
Their results in Figure 2 include experiments and simulations with the commercial software COMSOL, so we sought to both validate \pygbe using their experimental results, and replicate their computational findings. 
Again, we lack access to the data behind the plots, and we had to digitize the curves by hand. 
The quantity plotted in the original figures is reflectance as a function of wavenumber, whereas we compute the extinction---on the figures, they show inverted peaks, where we show positive peaks.
We can compare the results, nevertheless, because the quantity of interest is the wavenumber position of the peaks.

Challenges: all what's on the replication list that should go in the background 

Regarding Ellis:

Replication of Fig S4 supplementary material, was achieved successfully. 
Figure 2a validation and replication, achieved after correction for coupling. 
Challenges to validate with more results on the paper, since we did not count with enough 
information for the non-coupling case. So far we haven't attempted to run multiple pillars 
(ACTUALLY WE DID BUT I HAD THE WRONG NORMALS ON THE MESHES SO THE RESULTS WHERE TRASH), it is possible although we can not
introduce a substrate and have periodic boundary conditions.  


In general:
- How hard is to replicate when you do not count with enough information regarding software, methods, mesh discretization, etc. 
- Challenges to access data, therefore relying on digitalization of plot that introduce a lot of human error, specially if the quality 
of the plot is bad like in Rockstuhl et al. 
- Validation can be complicated when there are no standards/benchmarks on the field.
- We can say our validation and replication studies, our reproducible. So hopefully this can be use as
 an example on the field. 

\bibliographystyle{unsrt}
\bibliography{pygbe_rep_val} %don't leave spaces between elements, it throws error


\end{document}