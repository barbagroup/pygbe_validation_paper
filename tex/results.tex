%!TEX root = ClementiBarba2020.tex

\subsection{Replication of results from Rockstuhl, et al., 2005}

%%% Rockstuhl work summary%%%%
The work of Rockstuhl et al.\cite{rockstuhl2005} study the phonon-polariton response of silicon carbide (SiC)
nanoparticles using the boundary element method. They analyze different "cylindrical particles"
made of 6H-SiC as a function of geometrical cross section. These "cylinders" have an infinite
extension in the third dimension, and they solve these geometries numerically with a 2D 
boundary element method from their previous work \cite{rockstuhl2003}. 
%%% Rockstuhl work summary%%%%


We decided to replicate one of the results that they present on Fig.14 of their paper, which 
presents the scattering cross section of a SiC rectangular cylinder for three different object 
sizes. To comply with our quasistatic approach ($\lambda > d$ where $d$) we chose the particular 
case where a=672 nm and b=328 nm.

\textbf{Difference between methods, mesh and dielectric data}

\textit{method}: The main difference between Rockstuhl simulations and ours is that they solve a 2D problem, 
while we solve a 3D model. In our case the geometry has a finite third dimension. To cope with this
we will extend the third direction considerably, and study this effect.
We compute extinction cross section (scattering plus absorption) while Rockstuhl work present only scattering. 
But since in both scattering and absorption the resonance occur at the same wavelengths, the results are 
comparable.

\textit{mesh}: Their work does not provide any details regarding the discretization of their geometries or 
parameters involved in the simulations. We performed a grid independence study to make sure that we are
solving properly the physics and minimizing errors due to discretization. 

\textit{dielectric data}: The use 6H-SiC as material, and they obtained their data from a a source that we
were not able to replicate. Instead we are using experimental data for 4h-SiC that was provided to us 
via private communications.  

\subsubsection{Grid independance study}

We performed a grid independence study on a SiC cube of side $L=535$ nm submerged in air, under a 
constant electric field aligned with the z-axis.
(similar to quadratic cylinder on Fig 18 of Rockstuhl et.al). 
Due to the nature of the geometry, and its sharp edges it was challenging to see proper convergence. However,
for this type of physics, convergence has been studied in our previous work \cite{ClementiCooperBarba2019}. 
In the Figure \ref{fig:cube535} we show a grid independence study, where we go from a mesh of 15552 
triangles (density = 1.11x10$^-4$ $N/\text{\AA}$) to 19200 (density = 9.05x10$^-5$ $N/\text{\AA}$) 
triangles and the computed results do not change.

\begin{figure}
    \centering
    \includegraphics[width=0.85\textwidth]{cubeL535nm_15Kvs19K.pdf} 
    \caption{Grid independence study for a SiC cube of side $L=535 nm$ submerged in air under a constant 
    electric field in the z-direction. The curves represent the extinction cross section divided by $L^2$ 
    as a function of wavelength divided by $L$}
    \label{fig:cube535}
 \end{figure}

It is worth noting that the extinction cross section curve in Figure \ref{fig:cube535} has extra peaks, 
compared to the results of Figure 18 on Rockstuhl, this is due to the 3D nature of our case and the sharp 
edges. This will be approach in the following results where we attempt a replication of one of teh results 
of Figure 14 of Rockstuhl work.

\subsubsection{Replication of Figure 14 (case a1) of Rockstuhl 2005}

We chose to replicate the result of Rocksuthl et al. presented in Figure 14. In particular the case where $a=672$ nm 
and $b=328$ nmn, since these dimension comply with the requirements of the quasistatic approximation.
They present the normalized scattering cross section of a SiC rectangular cylinder, and they perform simulation 
for two different setups. In Rockstuhl Figure 14 (left) they have the wave vector (illumination) along the long 
side of the geometry, which means that the electric field is parallel to the short side of the rectangle. Following, 
a similar analysis Rockstuhl Figure 14 (right) they have the wave vector (illumination) along the short
side of the geometry, which means that the electric field is parallel to the long side of the rectangle. These orientations,
correspond to the images (A) and (B) in the sketch presented in Figure \ref{fig:rectangle_sketch}.

Based on the results of the grid independence stud we decided to we used the density of the cube as a reference to 
produce the meshes for the replication of figures 14a and 14b on Rockstuhl's work (case a1=672, b=328). For the third 
dimension we needed to chose a value that represents "infinity". To achieve this, we perform a study on the effects of 
elongate the third dimension. In Figure \ref{fig:ext_y_14} we present the results for two different values of the 
third dimension, $y=1344$ nm (2xa) and $y=2688$ nm (4xa). 
We see that as we extend the third dimension the intensity of some peaks decreases, this is because some
of the peaks are related to the third dimension. Therefore, from now on we use $y=y=2688$ nm. It is worth 
noting that we could have increased this value more, however, the simulations will become computational
more expensive and we will still have the effects of having a 3D model.

\begin{figure}
    \centering
    \includegraphics[width=0.45\textwidth]{rockstuhl_rectangles.pdf} 
    \caption{Rockstuhl runs configurations}
    \label{fig:rectangle_sketch}
\end{figure}

\begin{figure}
    \centering
    \subfloat{\includegraphics[width=0.45\textwidth]{ext_y_14a.pdf}}
    \subfloat{\includegraphics[width=0.45\textwidth]{ext_y_14b.pdf}} 
    \caption{Effect of the elongation of the third direction ($y$) on the 
        extinction cross section of a rectangular prism of SiC of dimensions $a=672$ nm 
        and $b=328$ nm, submerged in air and under a constant electric field 
        parallel to the z-axis. The left plot corresponds to a configuration such that the electric 
        field is parallel to $b$ (configuration (A) on Figure \ref{fig:rectangle_sketch}), while the 
        right plot corresponds to a configuration such that the electric field is 
        parallel to $a$ (configuration (B) on Figure \ref{fig:rectangle_sketch}}
    \label{fig:ext_y_14}   
 \end{figure}


To generate the meshes for these simulations, we used the open source software Trimesh 
(\url{https://github.com/mikedh/trimesh}), but we realized that it was not producing a 
uniform mesh and that it was not possible to produce regular triangles with the functions 
available. To overcome this, we created our own mesh using python scripts, and therefore 
obtain uniform meshes. We wanted to study the effect of a uniform mesh as well as the effect
on rounding the edges. We wanted to round the edges since in Rockstuhl work this was mentioned 
as a factor that will introduce extra peaks on the response. We were not able to control the 
roundness as a function of arc of curvature or the dimensions of the rectangular prism, so we 
decided to use the default settings on Trimesh. Figure \ref{fig:tri_reg_round_14} shows the 
results on the effect of uniformity and roundness. You can see that the second peak is not
present in the green curve, which can be attributed to the effect of the roundness. This is 
consistent with the results on Rockstuhl's work. 

\begin{figure}
    \centering
    \subfloat{\includegraphics[width=0.45\textwidth]{tri_reg_round_14a.pdf}}
    \subfloat{\includegraphics[width=0.45\textwidth]{tri_reg_round_14b.pdf}}
    \caption{ Effect of uniformity and roundness on the mesh on the 
    extinction cross section of a rectangular prism of SiC of dimensions $a=672$ nm, 
    $b=328$ nm and $y=2688$ nm, submerged in air and under a constant electric field 
    parallel to the z-axis}
    \label{fig:tri_reg_round_14}
 \end{figure}

Once we have found the "best" possible geometry approximation, we show how our results 
(green curve in Figure \ref{fig:tri_reg_round_14}) compare 
with the original results from Rockstuhl's Figure 14. We have obtained Rockstuhl's curves by using 
the WebPlotDigitizer (\url{https://apps.automeris.io/wpd/}). The replication results are 
presented in Figure \ref{fig:rep_14}. We can say that the main peaks presented in Rockstuhl et al. are 
successfully replicated. While we still have the presence of a third peak in our results, we believe 
this is a consequence of the 3D nature of our geometry.

 \begin{figure}
    \centering
    \subfloat{\includegraphics[width=0.45\textwidth]{replication_14a.pdf}}
    \subfloat{\includegraphics[width=0.45\textwidth]{replication_14b.pdf}} 
    \caption{Replication of Figure 14 from Rockstuhl 2005. Extinction cross section of a
    rectangular prism of SiC of dimensions $a=672$ nm, $b=328$ nm and $y=2688$ nm, submerged
    in air and under a constant electric field parallel to the z-axis.}
    \label{fig:rep_14}
 \end{figure}



 \subsection{Replication of results from Ellis et al., 2016, and validation}

There are two results from this paper that we aim to replicate, and one of them is a comparison
against experimental results which will lead to a validation. 

The first result we show is a replication of a result presented in their supplementary 
material FigureS4 {\color{red}(note sure how to cite this, maybe a link)}. We aimed to replicate the 
black curve since the simulation setup is possible to replicate using PyGBe. For each aspect ratio (AR) 
value from 1 to 7 we computed the extinction cross section $C_{ext}$ across the wave number in the range
800-1000 $cm^{-1}$, we identify the the lower frequency mode ($E_{100}$ in Ellis paper)  that is not a 
longitudinal mode. We identify the Longitudinal modes by comparing normal vs 22 deg incidence runs. These
modes due to its nature do not show up in the normal incidence runs. Then, on the 22 deg incidence 
computations we choose the lower wave number mode that it's not a longitudinal mode. The simulations were 
performed for the Long Edge orientation, meaning that the electric field is polarized along the long edge 
of the pillar that is not the height. 

\begin{figure}
    \centering
    \includegraphics[width=0.85\textwidth]{AR_22_vs_norm.pdf} 
    \caption{ Extinction cross section across wave number for SiC pillars of multiple Aspect ratios. 
             (H=950 nm, W=400nm, L=400-2800 (AR=1-7))
            Normalized by maximum extinction cross section as a function of wave number,
            for a pillar orientation such that the electric field is parallel polarized 
            (I HAVE TO EXPLAIN THIS BETTER, IT ALIGNS WITH THE LENGTH OF THE PILLAR). 
            We present the results for normal incidence and 22 degrees incidence, 
            for different aspect ratios (AR)
            }
    \label{fig:AR_22_vs_norm}
 \end{figure}


\begin{table}
    \begin{center}
      \caption{Wavelength at which peaks happen for different aspect ratios, for runs where the electric
      field is parallel to the length (L) of the pillar. We have normal incidence and 22 degrees.}
      \label{tab:table1}
      \begin{tabular}{c c c c c c c c}
        \textbf{AR} \\
        \hline
        \multirow{2}{*}{1} & $\perp$ & \textbf{917.73} & 934.092 & 949.604 & 957.325 \\ % <-- Combining 2 rows with arbitrary with (*) and content 12
        & 22$^{\circ}$ & 883.926 & \textbf{917.73} & 935.052 & 949.604 & 957.325 \\ % <-- Content of first column omitted.
        \hline
        \multirow{2}{*}{2} & $\perp$ & \textbf{903.233} & 926.395 & 944.762 & 958.242 \\ % <-- Combining 2 rows with arbitrary with (*) and content 12
        & 22$^{\circ}$ & 896.517 & \textbf{903.233} & 926.395 & 944.762 & 958.242 \\ % <-- Content of first column omitted.
        \hline
        \multirow{2}{*}{3} & $\perp$ & \textbf{888.793} & 922.552 & 931.223 & 948.613 & 958.242 \\ % <-- Combining 2 rows with arbitrary with (*) and content 12
        & 22$^{\circ}$ & \textbf{888.793} & 899.418 & 922.552 & 931.223 & 958.242 \\ % <-- Content of first column omitted.
        \hline
        \multirow{2}{*}{4} & $\perp$ & \textbf{876.186} & 929.32 & 946.639 & 958.242 \\ % <-- Combining 2 rows with arbitrary with (*) and content 12
        & 22$^{\circ}$ & \textbf{876.186} & 901.281 & 921.618 & 929.32 & 945.745 & 958.242 \\ % <-- Content of first column omitted.
        \hline
        \multirow{2}{*}{5} & $\perp$ & \textbf{865.576} & 926.395 & 945.745 & 958.242 \\ % <-- Combining 2 rows with arbitrary with (*) and content 12
        & 22$^{\circ}$ & \textbf{865.576} & 901.281 & 921.618 & 958.242 \\ % <-- Content of first column omitted.
        \hline
        \multirow{2}{*}{6} & $\perp$ &  \textbf{856.904} & 914.793 & 923.489 & 929.32 & 946.639 & 958.242\\ % <-- Combining 2 rows with arbitrary with (*) and content 12
        & 22$^{\circ}$ & \textbf{856.904} & 901.281 & 920.6 & 958.242\\ % <-- Content of first column omitted.
        \hline
        \multirow{2}{*}{7} & $\perp$ &  \textbf{850.134} & 910.963 & 921.618 & 928.372 & 946.639 & 958.242 \\ % <-- Combining 2 rows with arbitrary with (*) and content 12
        & 22$^{\circ}$ & \textbf{850.134} & 901.281 & 910.963 & 920.6 & 958.242\\ % <-- Content of first column omitted.
        \hline

      \end{tabular}
    \end{center}
  \end{table}


In other words, we choose the lower mode (lower wavenumber) from the normal 
incidence cases.

Our simulations differ on the method and we do not have a substrate included. The black curve on 
Ellis et al \cite{ellis2016} represents the resonance positions for the $E_{100}$ mode for the different 
aspect ratios when the pillar are separated by a 5000 nm gap. The simulations performed with \pygbe 
consist of one isolated pillar.

\begin{figure}
    \centering
    \includegraphics[width=0.85\textwidth]{AR_rep_FS4_Ellis2016.pdf} 
    \caption{Replication of figure S4 of supplementary material of Ellis 2016}
    \label{fig:rep_FS4_ellis}
 \end{figure}

 If we calculate de relative percentage error for each of these points we get:
 
 \begin{table}
    \centering
    \caption{\label{table:err_AR} Percentage error for different aspect ratios.} 
    \begin{tabular}{c c}
    \hline%\toprule
    ARR & \% error \\
    \hline%\midrule
     $1$ & $0.95$ \\
     $2$ & $0.67$ \\
     $3$ & $0.35$ \\
     $4$ & $0.16$ \\
     $5$ & $0.72$ \\
     $6$ & $1.20$ \\
     $7$ & $1.59$ \\
    \hline%\bottomrule
    \end{tabular}
\end{table}


\pygbe AR=4 attempt of validation Figure 2a (see Figure \ref{fig:pygbe_vs_exp_2a})

\begin{figure}
    \centering
    \includegraphics[width=0.85\textwidth]{pygbe_vs_exp_fig2a_Ellis.pdf} 
    \caption{\pygbe vs experiments, figure 2a of Ellis 2016. Ellis et al 
    has near field interaction in its experiments. Ellis data was digitized
    with web digitizer.}
    \label{fig:pygbe_vs_exp_2a}
 \end{figure}

First order approximation to achieve validation. In our code we do not have near 
field interaction. From Figure S4 in supplementary material of Ellis et al. We know that 
for the first mode ($E_{100}$) the difference between interaction and no interaction is 
12.17 $cm^{-1}$ (obtained from digitized data). If we use this as a first order approximation,
we can subtract that value from our curve, and the results are in Figure \ref{fig:val_2a}

\begin{figure}
    \centering
    \includegraphics[width=0.85\textwidth]{validation_FOA_fig2a_Ellis.pdf} 
    \caption{Validation against experiments in figure 2a of Ellis 2016, using first order approximation}
    \label{fig:val_2a}
 \end{figure}

If we use the same approximation and compare the results with Ellis simulations on
Figure 2 of their paper (green curve) we get (see Figure \ref{fig:rep_2a})

\begin{figure}
    \centering
    \includegraphics[width=0.85\textwidth]{replication_FOA_fig2a_Ellis.pdf} 
    \caption{Replication of simulations in figure 2a of Ellis 2016, using first
     order approximation}
    \label{fig:rep_2a}
 \end{figure}