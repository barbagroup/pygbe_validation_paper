%!TEX root = ClementiBarba2020.tex

We have presented several results that replicate previous published findings in the general area of nanostructure responses to electromagnetic waves. 
Our field of interest is computational nanoplasmonics for applications in biosensors, and in a previous publication we developed the mathematical formulation and reported both solution verification activities, and an application demo with our software \pygbe, extended to treat complex dielectrics and imposed electric fields \cite{ClementiETal2019}.
The search for a physical context and published results that would allow us to undertake validation studies with \pygbe is what led us to this work. 
Even if we finally do have validation and replication cases, neatly presented here, the path to obtain these results was nonlinear, iterative, and arduous.

In the first case, we sought to replicate a result from Rockstuhl et al., 2005 \cite{rockstuhl2005}, where they computed the scattering cross-section as a function of wavelength for a silicon carbide rectangular nanostructure. 
They present their results as a plot (Figure 14 in their paper), and report the numeric value of the resonance wavelength in the text. 
Lacking access to the secondary data behind the plots---computed from two-dimensional simulations with a boundary element solver---we were forced to manually digitize the values from the figure.
Our results are presented in Figure \ref{fig:rep_14}, together with the curve we obtain from digitizing the source image. 
We were successful at replicating the strong peaks reported  by Rockstuhl et al.\ at wavelengths
10.42 $\mu m$ and 10.7 $\mu m$,  when the electric field $E$ is parallel to short side of the rectangle, and 10.42 $\mu m$ and 10.82 $\mu m$
when $E$ is parallel to the long side. Our results contain extra (small) peaks that are not present in the work of Rockstul et al.
The first one, located between the main two peaks, we attribute to the the effect of
sharp edges (see Figure \ref{fig:tri_reg_round_14}). The second extra peak is the far right one, and we believe this
peak is a consequence of the 3D nature of our geometry; as observed in Figure \ref{fig:ext_y_14} this peak intensity
decreases as the third dimension of the geometry extends.

Regarding Rockstuhl:

- The far right peak is probably because of 3D EFFECTS, the second peak, in the original runs, it was definitely related to the the 
sharp edges since when we introduce a level of roundness, it disappear. The peaks wavelength match perfectly, as seen in the Figure 
\ref{fig:rep_14}. We can say we replicate this result. 
Challenges: all what's on the replication list that should go in the background 

Regarding Ellis:

Replication of Fig S4 supplementary material, was achieved successfully. 
Figure 2a validation and replication, achieved after correction for coupling. 
Challenges to validate with more results on the paper, since we did not count with enough 
information for the non-coupling case. So far we haven't attempted to run multiple pillars 
(ACTUALLY WE DID BUT I HAD THE WRONG NORMALS ON THE MESHES SO THE RESULTS WHERE TRASH), it is possible although we can not
introduce a substrate and have periodic boundary conditions.  


In general:
- How hard is to replicate when you do not count with enough information regarding software, methods, mesh discretization, etc. 
- Challenges to access data, therefore relying on digitalization of plot that introduce a lot of human error, specially if the quality 
of the plot is bad like in Rockstuhl et al. 
- Validation can be complicated when there are no standards/benchmarks on the field.
- We can say our validation and replication studies, our reproducible. So hopefully this can be use as
 an example on the field. 