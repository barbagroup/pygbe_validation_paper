Discuss points (still need to check results, to see we haven't left some of this there):

Regarding Rockstuhl:

- The far right peak is probably because of 3D EFFECTS, the second peak, in the original runs, it was definitely related to the the 
sharp edges since when we introduce a level of roundness, it disappear. The peaks wavelength match perfectly, as seen in the Figure 
\ref{fig:rep_14}. We can say we replicate this result. 
Challenges: all what's on the replication list that should go in the background 

Regarding Ellis:

Replication of Fig S4 supplementary material, was achieved successfully. 
Figure 2a validation and replication, achieved after correction for coupling. 
Challenges to validate with more results on the paper, since we did not count with enough 
information for the non-coupling case. So far we haven't attempted to run multiple pillars 
(ACTUALLY WE DID BUT I HAD THE WRONG NORMALS ON THE MESHES SO THE RESULTS WHERE TRASH), it is possible although we can not
introduce a substrate and have periodic boundary conditions.  


In general:
- How hard is to replicate when you do not count with enough information regarding software, methods, mesh discretization, etc. 
- Challenges to access data, therefore relying on digitalization of plot that introduce a lot of human error, specially if the quality 
of the plot is bad like in Rockstuhl et al. 
- Validation can be complicated when there are no standards/benchmarks on the field.
- We can say our validation and replication studies, our reproducible. So hopefully this can be use as
 an example on the field. 